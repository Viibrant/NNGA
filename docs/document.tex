\documentclass[conference]{inc/IEEEtran}
\IEEEoverridecommandlockouts
% The preceding line is only needed to identify funding in the first footnote. If that is unneeded, please comment it out.
% !TeX root = ../main.tex
\usepackage{a4wide}

\usepackage[utf8]{inputenc}

%\usepackage[ngerman]{babel}
\usepackage[english]{babel}

\usepackage[T1]{fontenc}
\usepackage{palatino}
\usepackage{cite}
\usepackage{graphicx}
\usepackage{caption}
\usepackage{url}
\usepackage{acronym}
\usepackage{tocloft}
\usepackage{algorithmic}
\usepackage{mathpazo}
\usepackage{amsmath}
\usepackage{amsfonts}
\usepackage{adjustbox}
\usepackage{textcomp}
%\usepackage{subcaption}

\usepackage{hhline}
\usepackage{fancyhdr}
\usepackage{amssymb}
\usepackage{floatflt}
\usepackage{setspace}
\usepackage{float}
\usepackage{booktabs}
\usepackage{color}
\usepackage{listings}
\usepackage{array}
\usepackage{scrhack}
\usepackage{xcolor}
\usepackage{wrapfig}
\usepackage[hidelinks]{hyperref}
\usepackage{url}
\usepackage{lmodern}
\usepackage{multirow}
\usepackage{subfig}
\usepackage{cleveref}
\usepackage{lipsum}
\usepackage[bottom, hang]{footmisc}


\def\BibTeX{{\rm B\kern-.05em{\sc i\kern-.025em b}\kern-.08em
    T\kern-.1667em\lower.7ex\hbox{E}\kern-.125emX}}
\begin{document}

\title{Neural Network-Based Iris Classification and Genetic Algorithm-Driven SHUBERT Function Optimisation \\
{\footnotesize Completed as part of \textit{M25352 Neural Networks and Genetic Algorithms}}
}


\author{\IEEEauthorblockN{Connor Brook}
    \IEEEauthorblockA{\textit{BSc. Data Science and Analytics} \\
        \textit{University of Portsmouth}\\
        Portsmouth, United Kingdom \\
        brook@connordata.science}
    \and
    \IEEEauthorblockN{Jamie Doe}
    \IEEEauthorblockA{\textit{BSc. Software Engineering} \\
        \textit{University of Portsmouth}\\
        Portsmouth, United Kingdom \\
        UP953068@myport.ac.uk}
}

\maketitle

\begin{abstract}
    In this paper, we explore two different optimisation problems: a neural network for the Iris dataset and a genetic algorithm for the SHUBERT function. We design, implement, and evaluate both optimisation techniques, comparing their performance and discussing the advantages and disadvantages of each method. The artificial neural network is employed to classify iris flowers based on their morphological features, while the genetic algorithm is utilized to optimize the SHUBERT function, a well-known benchmark problem in the global optimisation domain. By comparing the results of these two distinct optimisation approaches, we provide insights into their applicability and effectiveness in solving different types of problems.
\end{abstract}

\begin{IEEEkeywords}
    component, formatting, style, styling, insert
\end{IEEEkeywords}

\section{Introduction}
In recent years, optimisation techniques have gained significant attention due to their wide range of applications in various fields such as machine learning, engineering, and finance. Among these techniques, artificial neural networks (ANNs) and genetic algorithms (GAs) have emerged as powerful tools for solving complex problems. This paper explores the application of ANNs to classify the Iris dataset, a well-known benchmark problem in machine learning, and the use of GAs to optimize the SHUBERT function, a challenging global optimisation problem. By comparing the performance of these two distinct optimisation methods, we aim to provide a deeper understanding of their strengths and weaknesses, as well as their suitability for different types of problems. This paper will cover the design, implementation, evaluation, and comparison of ANNs and GAs in the context of the Iris dataset classification and SHUBERT function optimization.

\section{Part I: Neural Networks for Classification/Mapping Task}

\subsection{Introduction}

The Iris dataset is a classic machine learning classification problem, introduced in 1936 by Ronald A.
Fisher. This report aims to create an artificial neural network to accurately classify Iris plants into
three species: Iris Setosa, Iris Versicolour, and Iris Virginica. We tackle the challenge of non-linear
separability between two species through data preprocessing, neural network design, optimization,validation,
and evaluation. Our goal is to provide insights and pave the way for future research in this field.

\subsection{Data Analysis and Pre-processing}

\subsubsection{Data Analysis}
Before the ANN's architecture can be theorised, we first must perform a comprehensive data analysis of the
Iris dataset using various visualization techniques. This will allow us to gain a deeper understanding of the
relationships between the attributes and the distribution of the different Iris species.

\begin{itemize}
  \item \textbf{Boxplot}: We created a boxplot to visualize the distribution of each attribute (sepal length, sepal width,
  petal length, and petal width) across the three Iris species. This plot helped us identify any potential outliers
  and examine the overall spread of the data.
  \item \textbf{Pairplot}: A pairplot was generated to illustrate the relationships between all the possible pairs of attributes.
  This scatterplot matrix allowed us to observe the linear separability of Iris Setosa and the evident overlap between
  Iris Versicolour and Iris Virginica.

  The pairplot reveals that Iris Versicolour and Iris Virginica exhibit overlapping attributes, suggesting that accurately
  classifying these two Iris species could present challenges.

  \item \textbf{Correlation Matrix}: We calculated the correlation matrix to quantify the linear relationships between the attributes.
  By using a heatmap, we could easily identify strong positive correlations between petal length and petal width, as well as
  between petal length and sepal length.

  \item \textbf{Principal Component Analysis (PCA)}: We performed PCA with three components to reduce the dimensionality of the dataset
  and project the data onto a lower-dimensional space. A 3D scatterplot was created to visualize the clustering of the Iris
  species in this reduced space, providing insights into the separability of the classes.
  
  The findings demonstrated that Iris Versicolour and Iris Virginica indeed have overlapping attributes. However, upon
  incorporating an additional dimension, the overlap appeared less severe, indicating that with appropriate data preprocessing,
  these species could be classified.
\end{itemize}

\subsubsection{Data Pre-processing}
During the data processing stage, we perform a series of transformations to ensure that the data is properly prepared
for the neural network training. These transformations include:
\begin{itemize}
    \item \textbf{Scaling}: To avoid a scenario where features with larger values dominate the objective function, we scale the data to
    have zero mean and unit variance. This standardization process helps improve the neural network's training efficiency and
    overall performance.
    
    \item \textbf{One-hot encoding}: We convert categorical features into binary features through one-hot encoding. This transformation
    ensures that the model does not interpret categorical features as ordinal, which could lead to inaccurate classifications.
    
    \item \textbf{Train-test split}: To evaluate the model's performance on unseen data, we divide the dataset into a training set and a
    test set. The training set is utilized for training the model, while the test set serves as a benchmark for assessing the
    model's generalization capabilities.
\end{itemize}

By applying these transformations to the iris datasey, we can effectively preprocess the data, enabling the neural
network to learn the underlying patterns and relationships more efficiently. This, in turn, will lead to more accurate and
reliable classification results.
\subsection{Neural Network Design}
\subsubsection{Topology and Architecture}

The input layer comprises of four nodes, corresponding to the four distinct attributes of the Iris dataset: sepal length, sepal width,
petal length, and petal width. Following the input layer, we have five hidden layers, four containing eight nodes and one with four nodes.
The output layer consists of three nodes, representing the three Iris species we aim to classify: Iris Setosa, Iris Versicolour, and Iris Virginica.

This architecture was selected due to its potential for intricate pattern detection. By increasing the number of hidden layers and nodes within each
layer, the network can learn more complex representations of the data, thereby improving its ability to make accurate classifications. However, it's
important to note that adding more layers and nodes also increases the risk of overfitting. Thus, careful monitoring during training and validation
is necessary to prevent the model from memorizing the training data too closely and failing to generalize well to unseen data.

To facilitate learning within our network, we've chosen to use the Gaussian Error Linear Unit (GELU) as our activation function. GELU has a higher
threshold for activation, meaning that it requires stronger evidence before firing a neuron. This property can help our network learn more meaningful,
sparser representations of the data, which can be beneficial for the classification task at hand.

\subsubsection{Training Algorithm (Backpropagation)}

\subsection{Training and Testing}

The training of our deep neural network model was executed using the backpropagation algorithm. Backpropagation is an integral method for optimising
the weights within the network, serving as a specific implementation of a larger group of algorithms known as gradient descent algorithms. The goal
of backpropagation is to minimize the error in the output layer by adjusting the weights in the hidden layers.

The backpropagation process begins with a forward pass through the network, starting from the input layer and proceeding to the output layer. This
stage generates an initial prediction using the current network's state. The difference between this prediction and the actual target value is then
calculated as an error through a loss function.

Subsequently, the algorithm performs a backward pass through the network. During this phase, the derivative of the error with respect to each weight
is computed. This essentially quantifies how much each weight contributes to the total error. The weights are then updated in a direction that
minimises the error. This procedure is repeated until the performance of the network plateaus or begins to decline, indicating a state of overfitting.

In our case, the backpropagation algorithm was utilized in tandem with the stochastic gradient descent (SGD) optimization algorithm, which is renowned
for its efficiency when handling large datasets. We also introduced a 'MaxEpochs' value of 500 and an 'InitialLearnRate' of 0.25 for the SGD, defining
the parameters of our training process.

The implementation of the backpropagation training algorithm was instrumental in effectively training our deep neural network model and achieving the
observed performance on the Iris dataset. Balancing the learning rate and the number of epochs was a critical aspect of ensuring the model's accuracy
and robustness.ural network could effectively learn from the Iris dataset and perform multi-class classification. Notably, the inclusion of dropout
layers and batch normalization in the network architecture also contributed to the effective learning of the model, preventing overfitting and ensuring
the model generalises well to unseen data.

\subsubsection{Cross-validation}

\subsection{Results and Post-Processing}

\section{Part II: Genetic Algorithms for Global Optimization Problem}

\subsection{Function Selection and Analysis}
\subsection{Genetic Algorithm Design}

\subsubsection{Representation of Initial Population}

\subsubsection{Fitness Function}

\subsubsection{Selection Method}

\subsubsection{Reproductive Operators}

\subsubsection{Stopping Criteria}

\subsubsection{Parameters and Constraints}
\subsection{GA Analysis of the Run}

\subsubsection{Selection Strategy}

\subsubsection{Comparison of Strategies}
\subsection{Results and Discussion}

\subsubsection{Premature Convergence}

\subsubsection{Maintaining Selection Pressure}

\subsubsection{Balancing Exploration and Exploitation}

\section{Conclusion}

\section*{References}


\bibliography{bib/IEEEabrv, bib/mybib}
\bibliographystyle{inc/IEEEtranS}


\end{document}